You can find the video summarizing this instructions in the following link:
https://nielsenenterprise.sharepoint.com/:v:/s/LatAmXZTemplate/EcJQrKt0vjhCoPEwI-wMWUMBkiSTPZ0UCRzoqcFH3QOlBg?e=bEHfZk

1. Make sure that your working directory has the following architecture:
    |-inputs
        |-Cells_Chars.csv
        |-Cells_LastPeriod.csv
        |-MBD_NumDist.csv
        |-MBD_TypeTarget.csv
        |-VUE_Impacts.csv
        |-VUE_SampleNSPC.csv
    |-outputs
    |-scripts
        |-XZTGenerator.py
        |-MDDCatCellImpacts.py 
        |-LatAm-TemplateGenerator.ipynb

    Note: Remeber that Cells_Chars.csv, MBD_TypeTarget.csv, (MBD_NumDist.csv 
            if multi-channel project) are user-defined files; and 
            Cells_LastPeriod.csv, VUE_SampleNSPC.csv, VUE_Impacts.csv (and 
            MBD_NumDist.csv if single-channel project) are VUE data.


2. Open LatAm-TemplateGenerator.ipynb in your preferred IDE or Jupyter Notebook.
    Make sure your using your current user directory, and the working directory
    is correctly set to the path where the files are located.


3. Make sure the params are set to your preferences.
    - nspc_param :       a parameter that the script will use to flag the cells that
                            exceeds by percentage this value.
    - xf_param:          a parameter that the script will use to flag the cells that
                            exceeds by percentage this value.
    - cell_cat_param:    a parameter that, in conjuction with cell_weight_param, 
                            the script will use to a cell importance if they exceed 
                            by percentage this value.
    - cell_weight_param: a parameter that, in conjuction with cell_cat_param, 
                            the script will use to a cell importance if they exceed 
                            by percentage this value.


4. Run the Jupyter cell.